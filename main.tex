%%%%%%%%%%%%%%%%%%%%%%%%%%%%% Define Article %%%%%%%%%%%%%%%%%%%%%%%%%%%%%%%%%%
\documentclass[11pt]{article}
%%%%%%%%%%%%%%%%%%%%%%%%%%%%%%%%%%%%%%%%%%%%%%%%%%%%%%%%%%%%%%%%%%%%%%%%%%%%%%%

%%%%%%%%%%%%%%%%%%%%%%%%%%%%% Using Packages %%%%%%%%%%%%%%%%%%%%%%%%%%%%%%%%%%
\usepackage[margin=1in]{geometry} % Ajusta los márgenes
\usepackage{amssymb}
\usepackage{amsmath}
\usepackage{amsthm}
\usepackage{empheq}
\usepackage{mdframed}
\usepackage{booktabs}
\usepackage{lipsum}
\usepackage{color}
\usepackage{psfrag}
\usepackage{pgfplots}
\usepackage{bm}
\usepackage[spanish]{babel}
\usepackage[style=apa,backend=biber,sorting=nyt,language=spanish]{biblatex}
\usepackage{csquotes}
\usepackage{setspace}
\usepackage{helvet} % Para usar fuente sans-serif que es similar a Arial
\renewcommand{\familydefault}{\sfdefault}
\usepackage{fancyhdr}
\usepackage{graphicx}
\usepackage[table]{xcolor} % Agrega esto en el preámbulo
\usepackage{array}
\usepackage{tablefootnote}
\usepackage{chemfig}
\usepackage{siunitx}
\usepackage{multicol}
%%%%%%%%%%%%%%%%%%%%%%%%%%%%%%%%%%%%%%%%%%%%%%%%%%%%%%%%%%%%%%%%%%%%%%%%%%%%%%%

% Other Settings

%%%%%%%%%%%%%%%%%%%%%%%%%% Page Setting %%%%%%%%%%%%%%%%%%%%%%%%%%%%%%%%%%%%%%%
\graphicspath{{img/}}
\addbibresource{bibliography.bib}
\pagestyle{fancy}
\fancyhf{} % Limpia encabezados y pies
\fancyhead[L]{\includegraphics[height=1.2cm]{img/Logo estancia.png}} % Imagen izquierda
\fancyhead[R]{\thepage\hspace{0.5em}\includegraphics[height=1.2cm]{img/logo-FESC.png}} % Número y luego imagen derecha
\fancyfoot[C]{}
\renewcommand{\headrulewidth}{0pt} % Sin línea en el encabezado
\renewcommand{\thepage}{\arabic{page}} % Números arábigos
\addto\captionsspanish{\renewcommand{\tablename}{Tabla}}
%%%%%%%%%%%%%%%%%%%%%%%%%% Define some useful colors %%%%%%%%%%%%%%%%%%%%%%%%%%
\definecolor{ocre}{RGB}{243,102,25}
\definecolor{mygray}{RGB}{243,243,244}
\definecolor{deepGreen}{RGB}{26,111,0}
\definecolor{shallowGreen}{RGB}{235,255,255}
\definecolor{deepBlue}{RGB}{61,124,222}
\definecolor{shallowBlue}{RGB}{235,249,255}
% Define los colores (puedes ajustar los valores RGB para afinar el tono)
\definecolor{unamBlue}{RGB}{0,56,168}      % Azul fuerte UNAM
\definecolor{unamLightBlue}{RGB}{153,180,225} % Azul claro UNAM
%%%%%%%%%%%%%%%%%%%%%%%%%%%%%%%%%%%%%%%%%%%%%%%%%%%%%%%%%%%%%%%%%%%%%%%%%%%%%%%

%%%%%%%%%%%%%%%%%%%%%%%%%% Define an orangebox command %%%%%%%%%%%%%%%%%%%%%%%%
\newcommand\orangebox[1]{\fcolorbox{ocre}{mygray}{\hspace{1em}#1\hspace{1em}}}
%%%%%%%%%%%%%%%%%%%%%%%%%%%%%%%%%%%%%%%%%%%%%%%%%%%%%%%%%%%%%%%%%%%%%%%%%%%%%%%

%%%%%%%%%%%%%%%%%%%%%%%%%%%% English Environments %%%%%%%%%%%%%%%%%%%%%%%%%%%%%
\newtheoremstyle{mytheoremstyle}{3pt}{3pt}{\normalfont}{0cm}{\rmfamily\bfseries}{}{1em}{{\color{black}\thmname{#1}~\thmnumber{#2}}\thmnote{\,--\,#3}}
\newtheoremstyle{myproblemstyle}{3pt}{3pt}{\normalfont}{0cm}{\rmfamily\bfseries}{}{1em}{{\color{black}\thmname{#1}~\thmnumber{#2}}\thmnote{\,--\,#3}}
\theoremstyle{mytheoremstyle}
\newmdtheoremenv[linewidth=1pt,backgroundcolor=shallowGreen,linecolor=deepGreen,leftmargin=0pt,innerleftmargin=20pt,innerrightmargin=20pt,]{theorem}{Theorem}[section]
\theoremstyle{mytheoremstyle}
\newmdtheoremenv[linewidth=1pt,backgroundcolor=shallowBlue,linecolor=deepBlue,leftmargin=0pt,innerleftmargin=20pt,innerrightmargin=20pt,]{definition}{Definition}[section]
\theoremstyle{myproblemstyle}
\newmdtheoremenv[linecolor=black,leftmargin=0pt,innerleftmargin=10pt,innerrightmargin=10pt,]{problem}{Problem}[section]
%%%%%%%%%%%%%%%%%%%%%%%%%%%%%%%%%%%%%%%%%%%%%%%%%%%%%%%%%%%%%%%%%%%%%%%%%%%%%%%

%%%%%%%%%%%%%%%%%%%%%%%%%%%%%%% Plotting Settings %%%%%%%%%%%%%%%%%%%%%%%%%%%%%
\usepgfplotslibrary{colorbrewer}
\pgfplotsset{width=8cm,compat=1.9}
%%%%%%%%%%%%%%%%%%%%%%%%%%%%%%%%%%%%%%%%%%%%%%%%%%%%%%%%%%%%%%%%%%%%%%%%%%%%%%%

%%%%%%%%%%%%%%%%%%%%%%%%%%%%%%% Title & Author %%%%%%%%%%%%%%%%%%%%%%%%%%%%%%%%
\title{Desarrollo de tabletas de trans-resveratrol y la
cuantificación de su principio activo a través de
espectrofotometría UV-Vis}
\author{García Benítez Emanuel Tonatiuh}
\date{Colegio de Ciencias y Humanidades\\
        Universidad Nacional Autónoma de México}
%%%%%%%%%%%%%%%%%%%%%%%%%%%%%%%%%%%%%%%%%%%%%%%%%%%%%%%%%%%%%%%%%%%%%%%%%%%%%%%

\begin{document}
\onehalfspacing

% Portada (sin numeración)
\begin{titlepage}
    \thispagestyle{empty}
    \vfill % Espacio flexible arriba

    % Dos imágenes centradas antes del texto y minipages
    \begin{center}
        \includegraphics[height=3cm]{img/Logo estancia.png}
        \hspace{2cm}
    \end{center}

    \vspace{1.5cm} % Espacio entre las imágenes nuevas y el resto

    \begin{center}
        {\LARGE \textbf{Desarrollo de tabletas de trans-resveratrol y la cuantificación de su principio activo a través de espectrofotometría UV-Vis}}
        \vspace{1cm}

        \noindent
        \begin{minipage}[c]{0.2\textwidth}
            \centering
            \includegraphics[width=3cm]{img/cch.png}
        \end{minipage}
        \hfill
        \begin{minipage}[c]{0.55\textwidth}
            \centering
            {\large García Benítez Emanuel Tonatiuh}\\[0.5cm]
            {\large Área: Ciencias Químico-Biológicas y de la Salud}\\[0.5cm]
            {\large Colegio de Ciencias y Humanidades}\\
            {\large Universidad Nacional Autónoma de México}\\
            {\large FES Cuautitlán}\\
        \end{minipage}
        \vspace{2cm}
        \hfill
        \begin{minipage}[c]{0.2\textwidth}
            \centering
            \includegraphics[width=3.0cm]{img/UNAM.jpg}
        \end{minipage}
        \includegraphics[height=3cm]{img/logo-FESC.png}
    \end{center}
    \vfill % Espacio flexible abajo
\end{titlepage}

\setcounter{page}{1} % Inicia numeración en la siguiente página

% El resto del documento
\section{Resumen}
    \input{lib/Resumen.tex}

\section{Introducción}
    El trans-resveratrol es un compuesto ampliamente utilizado como antioxidante debido
 a su capacidad para interactuar con radicales libres por los grupos -OH que 
 lo conforman. Este se encuentra en una amplia gama de alimentos y 
 existen distintos suplementos alimenticios que utilizan a este como principio activo.
 Sin embargo, estos, no han mostrado una eficacia
 significativa como factor preventivo o terapéutico en enfermedades.
 Para atender a esta problemática, se ha propuesto el desarrollo de tabletas
 que contienen un excipiente específico que logra que el trans-resveratrol
 se libere de forma controlada y sea completamente absorbido por las células
 del cuerpo, lo que podría mejorar su biodisponibilidad y eficacia.
 

\section{Planteamiento del problema}
    Los productos comerciales que existen actualmente en el mercado
 como \input{lib/apoyo/productostransresv.tex}que incluyen al trans-resveratrol como principio activo, no pueden garantizar
 la eficacia de sus efectos antioxidantes y terapéuticos, ya que no se ha
 logrado una formulación que permita su absorción efectiva en el organismo.
Por lo tanto, es necesario desarrollar una tableta con un excipiente específico que permita introducir
al mercado un producto que pueda garantizarse para uso preventivo de padecimientos
como el desarrollo de tumores cancerígenos en el organismo.

\section{Objetivos}
    \subsection{Objetivo general}
Crear tabletas de trans-resveratrol con un excipiente específico que pueda ayudar a permitir
 su liberación controlada y su completa absorción en el organismo, para mejorar
 su biodisponibilidad y eficacia como antioxidante.

\subsection{Objetivos específicos}
\begin{itemize}
    \item Implementar una formulación de excipiente para una tableta que contenga trans-resveratrol.
    \item Elaborar una curva de calibración con un estándar de trans-resveratrol con un %insertar pureza del compuesto
    pureza conocida.
    \item Hacer la cuantificación del trans-resveratrol por medio de espectrofotometría UV-Vis.
    \item Evaluar la calidad del método de cuantificación a través del análisis estadístico.
\end{itemize}

\section{Marco teórico}
    El trans-resveratrol ($C_2H_12O_3$) (\ref{anexo:estructura}) es un estilbeno natural (3,5,4'-trihidroxiestilbeno),
un tipo de polifenol no flavonoide. Esto quiere decir que 
 presente principalmente en la piel de las uvas rojas,
el vino tinto, el cacao, las mora, raíces de ciertas plantas como \textit{Polygonum cuspidatum}
 y algunos frutos secos.

\section{Metodología}
    Se utilizó un estándar de trans-resveratrol con un \num{80.32} \%
de pureza %insertar marca
, el cual se le realizó un barrido dentro de un espectrofotómetro (\ref{anexo:espectro}) UV-Vis marca %agregar marca
y se encontró que la longitud de onda máxima de absorción es de \num{305} y %ingresar la 2da longitud máxima
\SI{318}{\nano\meter}, lo cual corresponde con lo reportado por la literatura.

Para garantizar la repetibilidad del analista, se hicieron múltiples curvas de calibración por triplicados con las siguientes diluciones para cumplir 
los 5 puntos requeridos

A través de análisis de regresión lineal se garantizó la relevancia estadística de la curva de calibración, 
con los siguientes criterios (\ref{anexo:stats}):

\begin{multicols}{2}
    \begin{itemize}
        \item Coeficiente de variación: $\leq 3\%$
        \item Intercepción con eje $Y$: $\leq 0$
        \item Valor de $R^2$: $>99\%$
        \item Coeficiente de Pearson: $>99\%$
    \end{itemize}
\end{multicols}

Se realizó una dispersión para realizar las tabletas con trans-resveratrol. La formulación con los excipientes para el
del vehículo de la tableta fue la siguiente:

\begin{multicols}{2}
    \begin{itemize}
        \item Excipiente: 1.5 g
        \item Ex
    \end{itemize}
\end{multicols}

Una vez preparada la dispersión, se comprimieron las pastillas de trans-resveratrol
en una tableteadora monopunsante
con un peso de \SI{300}{\milli\gram} cada una. Se realizaron 30 tabletas en total,
las cuales se destinaron a las siguientes pruebas:

\begin{itemize}
    \item \textbf{Disolución:} Se realizó el molido de 10 tabletas  y se disolvieron en \SI{100}{\milli\liter} de metanol grado HPLC, 
    de donde se hizo una dilución más de \SI{2}{\milli\liter} y se dimidió la absorbancia a las longitudes de onda máximas encontradas.
    \item \textbf{Desintegración}: Se realizó una prueba de desintegración en agua destilada a temperatura ambiente.
    \item \textbf{Dureza:} Se midió la dureza de las tabletas utilizando un durómetro.
\end{itemize}

\section{Resultados}
    Los datos obtenidos en la curva de calibración fueron los siguientes (\ref{anexo:stats}):
% Insertar los datos reales
\begin{multicols}{2}
    \begin{itemize}
        \item Coeficiente de variación: $\leq 3\%$
        \item Intercepción con eje $Y$: $\leq 0$
        \item Valor de $R^2$: $>99\%$
        \item Coeficiente de Pearson: $>99\%$
    \end{itemize}
\end{multicols}

Los resultados obtenidos en la valoración de la cantidad de principio activo en las tabletas
fueron los siguientes (\ref{anexo:valoracionresult}):

En la prueba de desintegración, se obtuvo que el tiempo de desintegración de la tableta
superó lo normativo para una tableta de desintegración rápida.

En la prueba del durómetro se obtuvo un coeficiente de variación del %insertar CV
(\ref{anexo:durometroresult}).

Al realizar la prueba de friabilidad, se obtuvo un porcentaje de friabilidad del %insertar porcentaje
(\ref{anexo:friabilidad})



\section{Discusión}
    Por los resultados obtenidos en la curva de calibración, se puede decir que los resultados
son estadisticamente relevantes ya que se cumplen los estándares de calidad de %¿Dónde?

Los resultados de la valoración muestran que %¿Qué?

El tiempo de desintegración de la tableta pudo no haber cumplido con lo correspondiente
a una tableta de desintegración rápida por diversos factores, entre ellos se encuentran:
\begin{itemize}
    \item La formulación del vehículo no fue la adecuada.
    \item La compresión que se le hizo a las tabletas fue demasaida, teniendo que utilizar otro método de compresión que aplique menores presiones a cada tableta.
    \item La humedad del ambiente afectó la tableta.
    \item La temperatura en el desintegrador no fue la suficiente, teniendo que haber alcanzado los \SI{37}{\degreeCelsius} para cumplir con la norma.
\end{itemize}

Los resultados del durómetro muestran que %¿Qué?

Los resultados de la prueba de friabilidad muestran que %¿Qué?

\section{Conclusiones}
    \input{lib/conclusiones.tex}

\clearpage
\pagenumbering{gobble}

\printbibliography

% Hoja de anexos SIN numeración
\clearpage
\pagenumbering{gobble} % Detiene la numeración de página

\section*{Anexos}
\addcontentsline{toc}{section}{Anexos}

% Numeración de anexos tipo "Anexo 1", "Anexo 2", etc.
\newcounter{anexo}
\renewcommand{\theanexo}{Anexo \arabic{anexo}}

% Ejemplo de anexo numerado
\refstepcounter{anexo}
\subsection*{\theanexo. Estructura del trans-resveratrol}
\label{anexo:estructura}
 \begin{figure}[h!]
    \centering
    \includegraphics[width=0.5\textwidth]{trans-resveratrol.png}
    \caption{Estructura del trans-resveratrol}
    \label{fig:mi_imagen}
    \smilesobabel{Oc1ccc(/C=C/c2cc(O)cc(O)c2)cc1}{}
\end{figure}

\refstepcounter{anexo}
\subsection*{\theanexo. Datos físico-químicos del trans-resveratrol}
\begin{table}[h!]
    \centering
    \rowcolors{2}{white}{unamLightBlue!30}
    \begin{tabular}{|
        >{\centering\arraybackslash\cellcolor{unamBlue!80}\color{white}}m{4cm}
        |>{\centering\arraybackslash}m{7cm}
        |>{\centering\arraybackslash}m{4cm}|}
        \hline
        \rowcolor{unamBlue!90}
        {\cellcolor{unamBlue!90}\color{white}\textbf{Propiedad}} & {\color{white}\textbf{Valor aproximado}} & {\color{white}\textbf{Tomado de}} \\
        \hline
        \cellcolor{unamBlue!80}\color{white} Masa molar & \SI{228.24}{\gram\per\mole} & \\
        \cellcolor{unamBlue!60}\color{white} Solubilidad en agua & Muy baja (\SI{0.03}{\milli\gram\per\milli\liter}) & \\
        \cellcolor{unamBlue!80}\color{white} Solubilidad en etanol/metanol & Alta & \\
        \cellcolor{unamBlue!60}\color{white} LogP (coeficiente de partición) & \num{3.1} (moderadamente lipofílico) & \\
        \cellcolor{unamBlue!80}\color{white} pKa & \num{9.0} & \\
        \cellcolor{unamBlue!60}\color{white} Estabilidad & Sensible a luz UV y al $O_2$ (puede isomerizarse a cis) & \\
        \cellcolor{unamBlue!80}\color{white} Forma cristalina & Sólido blanco-cristalino & \\
        \hline
    \end{tabular}
    \caption{Propiedades fisico-químicas relevantes del trans-resveratrol.}
\end{table}
\label{propiedades}

\refstepcounter{anexo}
\subsection*{\theanexo. Estándar de trans-resveratrol.}
\label{anexo:estandar}

\refstepcounter{anexo}
\subsection*{\theanexo. Espectrofotometro UV-Vis marca.}
\label{anexo:espectro}

\refstepcounter{anexo}
\subsection*{\theanexo. Disoluciones para la curva de calibración.}
\label{anexo:disoluciones}

\refstepcounter{anexo}
\subsection*{\theanexo. Desintegrador marca}
\label{anexo:desintegrador}

\refstepcounter{anexo}
\subsection*{\theanexo. Durometro marca }
\label{anexo:durometro}

\refstepcounter{anexo}
\subsection*{\theanexo. Fragilizador marca}
\label{anexo:fragilizador}

\refstepcounter{anexo}
\subsection*{\theanexo. Datos de la curva de calibración.}
\label{anexo:stats}

\refstepcounter{anexo}
\subsection*{\theanexo. Resultados de la valoración}
\label{anexo:valoracionresult}

\refstepcounter{anexo}
\subsection*{\theanexo. Valores obtenidos en la prueba de dureza.}
\label{anexo:durometroresult}

\refstepcounter{anexo}
\subsection*{\theanexo. Datos obtenidos de la prueba de friabilidad}
\label{anexo:friabilidad}
% ...agrega más anexos repitiendo el bloque anterior

\end{document}