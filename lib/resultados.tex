Los datos obtenidos en la curva de calibración fueron los siguientes (\ref{anexo:stats}):
% Insertar los datos reales
\begin{multicols}{2}
    \begin{itemize}
        \item Coeficiente de variación: $\leq 3\%$
        \item Intercepción con eje $Y$: $\leq 0$
        \item Valor de $R^2$: $>99\%$
        \item Coeficiente de Pearson: $>99\%$
    \end{itemize}
\end{multicols}

Los resultados obtenidos en la valoración de la cantidad de principio activo en las tabletas
fueron los siguientes (\ref{anexo:valoracionresult}):

En la prueba de desintegración, se obtuvo que el tiempo de desintegración de la tableta
superó lo normativo para una tableta de desintegración rápida.

En la prueba del durómetro se obtuvo un coeficiente de variación del %insertar CV
(\ref{anexo:durometroresult}).

Al realizar la prueba de friabilidad, se obtuvo un porcentaje de friabilidad del %insertar porcentaje
(\ref{anexo:friabilidad})

