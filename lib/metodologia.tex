Se utilizó un estándar de trans-resveratrol con un \num{80.32} \%
de pureza %insertar marca
, el cual se le realizó un barrido dentro de un espectrofotómetro (\ref{anexo:espectro}) UV-Vis marca %agregar marca
y se encontró que la longitud de onda máxima de absorción es de \num{305} y %ingresar la 2da longitud máxima
\SI{318}{\nano\meter}, lo cual corresponde con lo reportado por la literatura.

Para garantizar la repetibilidad del analista, se hicieron múltiples curvas de calibración por triplicados con las siguientes diluciones para cumplir 
los 5 puntos requeridos (\ref{dilucionestable}):

\begin{table}[h!]
    \centering
    \rowcolors{2}{white}{unamLightBlue!30}
    \begin{tabular}{|>{\centering\arraybackslash\cellcolor{unamBlue!80}\color{white}}m{4cm}|>{\centering\arraybackslash}m{7cm}|}
        \hline
        \rowcolor{unamBlue!90}
        {\cellcolor{unamBlue!90}\color{white}\textbf{Solución}} & {\color{white}\textbf{Concentración}}\\
        \hline
        \cellcolor{unamBlue!60}\color{white} stock & \SI{20}{\micro\gram\per\milli\liter} \\
        \cellcolor{unamBlue!80}\color{white} \num{1} & \SI{4}{\micro\gram\per\milli\liter} \\
        \cellcolor{unamBlue!60}\color{white} \num{2} & \SI{8}{\micro\gram\per\milli\liter} \\
        \cellcolor{unamBlue!80}\color{white} \num{3} & \SI{12}{\micro\gram\per\milli\liter} \\
        \cellcolor{unamBlue!60}\color{white} \num{4} & \SI{16}{\micro\gram\per\milli\liter} \\
        \cellcolor{unamBlue!80}\color{white} \num{5} & \SI{20}{\micro\gram\per\milli\liter} \\
        \hline
    \end{tabular}
    \caption{Concentración de cada dilución de trans-resveratrol}
    \label{dilucionestable}
\end{table}

A través de análisis de regresión lineal se garantizó la relevancia estadística de la curva de calibración, 
con los siguientes criterios:

\begin{multicols}{2}
    \begin{itemize}
        \item Coeficiente de variación (CV): $\leq 3\%$
        \item Intercepción con eje $Y$: $\leq 0$
        \item Valor de $R^2$: $>99\%$
        \item Coeficiente de Pearson: $>99\%$
    \end{itemize}
\end{multicols}

Se realizó una dispersión para realizar las tabletas con trans-resveratrol. La formulación 
del vehículo fue con los siguientes excipientes:

\begin{multicols}{2}
    \begin{itemize}
        \item Excipiente: 1.5 g
    \end{itemize}
\end{multicols}

Una vez preparada la dispersión, se comprimieron las tabletas de trans-resveratrol
en una tableteadora monopunsante
con un peso de \SI{300}{\milli\gram} cada una. Se realizaron 30 tabletas en total,
las cuales se destinaron a las siguientes pruebas:

Para la prueba de valoración, se trituraron de 10 tabletas, de donde se tomaron %cantidad de gramos
del polvo y se disolvió en \SI{100}{\milli\liter} de metanol grado HPLC, 
de donde se hizo una dilución de \SI{2}{\milli\liter} de esta disolución
en \SI{25}{\milli\liter}. A partir de esta disolución final, se hizo la cuantificación
del principio activo que había en las tabletas midiendo su absorbancia a \SI{305}{\nano\meter}
en el espectrofotómetro UV-Vis
y utilizando la curva de calibración obtenida.

Se realizó una prueba de desintegración con una tableta en el desintegrador \ref{desintegrador} %ingresar marca
en agua corriente a temperatura de \SI{33}{\degreeCelsius}.

Se midió la dureza (\ref{anexo:durometro}) de las tabletas con 10 de estas, utilizando un durómetro %marca
%¿Es aceptable u homogenea?

Se utilizó un fragilizador (\ref{anexo:friabilidad}) %marca
para medir la friabilidad de las tabletas, en donde se utilizaron 6 de estas, obteniendo un porcentaje de friabilidad del %insertar porcentaje
de las tabletas.


