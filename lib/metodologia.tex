Se utilizó un estándar de trans-resveratrol con un \num{80.32} \%
de pureza %insertar marca
, el cual se le realizó un barrido dentro de un espectrofotómetro (\ref{anexo:espectro}) UV-Vis marca %agregar marca
y se encontró que la longitud de onda máxima de absorción es de \num{305} y %ingresar la 2da longitud máxima
\SI{318}{\nano\meter}, lo cual corresponde con lo reportado por la literatura.

Para garantizar la repetibilidad del analista, se hicieron múltiples curvas de calibración por triplicados con las siguientes diluciones para cumplir 
los 5 puntos requeridos

A través de análisis de regresión lineal se garantizó la relevancia estadística de la curva de calibración, 
con los siguientes criterios (\ref{anexo:stats}):

\begin{multicols}{2}
    \begin{itemize}
        \item Coeficiente de variación: $\leq 3\%$
        \item Intercepción con eje $Y$: $\leq 0$
        \item Valor de $R^2$: $>99\%$
        \item Coeficiente de Pearson: $>99\%$
    \end{itemize}
\end{multicols}

Se realizó una dispersión para realizar las tabletas con trans-resveratrol. La formulación con los excipientes para el
del vehículo de la tableta fue la siguiente:

\begin{multicols}{2}
    \begin{itemize}
        \item Excipiente: 1.5 g
        \item Ex
    \end{itemize}
\end{multicols}

Una vez preparada la dispersión, se comprimieron las pastillas de trans-resveratrol
en una tableteadora monopunsante
con un peso de \SI{300}{\milli\gram} cada una. Se realizaron 30 tabletas en total,
las cuales se destinaron a las siguientes pruebas:

\begin{itemize}
    \item \textbf{Disolución:} Se realizó el molido de 10 tabletas  y se disolvieron en \SI{100}{\milli\liter} de metanol grado HPLC, 
    de donde se hizo una dilución más de \SI{2}{\milli\liter} y se dimidió la absorbancia a las longitudes de onda máximas encontradas.
    \item \textbf{Desintegración}: Se realizó una prueba de desintegración en agua destilada a temperatura ambiente.
    \item \textbf{Dureza:} Se midió la dureza de las tabletas utilizando un durómetro.
\end{itemize}