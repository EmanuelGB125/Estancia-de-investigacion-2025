Se utilizó un estándar de trans-resveratrol con un %insertar pureza del compuesto
de pureza (\ref{anexo:estandar}), el cual se le realizó un barrido dentro de un espectrofotómetro (\ref{anexo:espectro}) UV-Vis marca %agregar marca
y se encontró que la longitud de onda máxima de absorción es de \num{305} y %ingresar la 2da longitud máxima
\SI{318}{\nano\meter}, lo cual corresponde con lo reportado por la literatura.

Se hicieron múltiples curvas de calibración (\ref{anexo:disoluciones}) con las siguientes diluciones para cumplir 
los 5 puntos requeridos:

\begin{itemize}
    \item \textbf{Concentración de la disolución stock:} \SI{20}{\micro\gram\per\milli\liter}
\end{itemize}
\begin{multicols}{3}
\begin{itemize}
    \item Disolución 1: \SI{4}{\micro\gram\per\milli\liter}
    \item Disolución 2: \SI{8}{\micro\gram\per\milli\liter}
    \item Disolución 3: \SI{12}{\micro\gram\per\milli\liter}
    \item Disolución 4: \SI{16}{\micro\gram\per\milli\liter}
    \item Disolución 5: \SI{20}{\micro\gram\per\milli\liter}
\end{itemize}
\end{multicols}

A través de análisis de regresión lineal se garantizó la repetibilidad del analista, 
garantizando los siguientes puntos (\ref{anexo:stats}):

\begin{multicols}{2}
    \begin{itemize}
        \item Coeficiente de variación: $\leq 3\%$
        \item Intercepción con eje $Y$: $\leq 0$
        \item Valor de $R^2$: $>99\%$
        \item Coeficiente de Pearson: $>99\%$
    \end{itemize}
\end{multicols}

Se realizó una dispersión para realizar las tabletas con trans-resveratrol. La formulación 
del excipiente fue la siguiente:

\begin{multicols}{2}
\begin{itemize}
    \item Excipiente: 1.5 g
\end{itemize}
\end{multicols}

Una vez preparada la dispersión, se comprimieron las pastillas de trans-resveratrol
en una tableteadora %revisar el nombre
con un peso de \SI{200}{\milli\gram} cada una. Se realizaron 22 tabletas