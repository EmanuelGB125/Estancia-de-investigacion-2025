El trans-resveratrol ($C_2H_12O_3$) (\ref{anexo:estructura}), isómero biológico activo del resveratrol,
 es un estilbeno\footnote{Un estilbeno es una molécula compuesta por dos anillos aromáticos conectados por un puente de carbono con un enlace doble} 
 (3,5,4'-trihidroxiestilbeno) natural,
un tipo de polifenol (con muchos grupos fenoles\footnote{Un fenol es una molécula que tiene un anillo aromático al que se le une un grupo hidroxilo})
no flavonoide (tipo de polifenol, pero con estructura diferente\footnote{Tiene tres anillos en vez de dos o uno de los anillos está dentro de un ciclo más grande (heterocíclico)} a los estilbenos).


Esta estructura molécular, le provee distintas propiedades al compuesto. Los fenoles le dan la capacidad
de formar puentes de hidrógeno, interactuar con otras moléculas y ser antióxidante \parencite{polifenolantioxidante};
el polifenol indica que tiene varios grupos $-OH$, lo que le da actividad biológica y la capacidad de
atrapar radicales libres; el estilbeno define su estructura rigida y lineal, con propiedades ópticas
y estabilidad estructural; y el que no sea flavonoide, lo diferencia de otras sustancias vegetales
con tres anillos, lo cual indica que requiere otras condiciones para su análisis y absorción.

Este se encuentra presente principalmente en la piel de las uvas rojas,
el vino tinto, el cacao, las moras, las raíces de ciertas plantas como \textit{Polygonum cuspidatum}
 y algunos frutos secos. Además, se ha comercializado en la industria de los suplementos
 alimenticios