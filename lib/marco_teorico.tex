El trans-resveratrol ($C_2H_12O_3$) (\ref{anexo:estructura}), isómero biológico activo del resveratrol,
 es un estilbeno\footnote{Un estilbeno es una molécula compuesta por dos anillos aromáticos conectados por un puente de carbono con un enlace doble} 
 (3,5,4'-trihidroxiestilbeno) natural,
un tipo de polifenol (con muchos grupos fenoles\footnote{Un fenol es una molécula que tiene un anillo aromático al que se le une un grupo hidroxilo})
\parencite{trans-resveratrol}.

Esta estructura molecular, le provee distintas propiedades al compuesto. Los fenoles le dan la capacidad
de formar puentes de hidrógeno, interactuar con otras moléculas y ser antioxidante \parencite{polifenolantioxidante};
el polifenol indica que tiene varios grupos $-OH$, lo que le da actividad biológica y la capacidad de
atrapar radicales libres; el estilbeno define su estructura rígida y lineal, con propiedades ópticas
y estabilidad estructural.

Estudios han demostrado que el resveratrol tiene la capacidad de retrasar tumores
\parencite{actividadbiologicaresveratrol}

Debido a sus características físico-químicas (\ref{propiedades}),
el trans-resveratrol es suceptible a su isomerización
a la forma cis, que es menos estable y menos activa biológicamente.
Esto puede ocurrir bajo condiciones de pH 

Este compuesto se encuentra presente principalmente en la piel de las uvas rojas,
el vino tinto, el cacao, las moras, las raíces de ciertas plantas como \textit{Polygonum cuspidatum}
 y algunos frutos secos. Además, se ha comercializado en la industria de los suplementos
 alimenticios